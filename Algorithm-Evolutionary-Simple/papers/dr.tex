%%% Local Variables: 
%%% mode: latex
%%% TeX-master: t
%%% End: 

\documentclass{article}
\usepackage[ansinew]{inputenc}
\usepackage{graphicx}
\usepackage{color}

\title{Experimentación con algoritmos distribuidos usando herramientas libres y gratuitas}

\author{Dr. Juan Julián Merelo Guervós,\\
Dra. Maribel García Arenas\\
Dr. Rodolfo García Bermúdez y \\
MSc. José Albert Cruz Almaguer 
}

\begin{document}

\maketitle

\begin{abstract}
En un entorno de restricción de costes para grandes instalaciones
computacionales acompañado de la existencia de herramientas en la nube
y ordenadores de sobremesa de altas capacidades, la experimentación
con algoritmos distribuidos se puede hacer fácilmente combinando ambas
cosas. En este trabajo se presenta una metodología de experimentación
con algoritmos genéticos distribuidos usando servicios de
almacenamiento en la nube tales como Dropbox (o alternativas libres
auto-instalables) y aplicaciones para gestión de máquinas virtuales
tales como VirtualBox. Usando el almacenamiento en la nube como un
sistema de intercambio de soluciones entre los diferentes nodos, se
tratará de probar la aplicabilidad de esta metodología así como probar
las capacidades de estos nuevos algoritmos evolutivos distribuidos. 
\end{abstract}

\section{Introducción}

La computación paralela no necesita ser complicada y prever escenarios
complicados o grandes variaciones de estructura de los programas
secuenciales. La mayor parte de los ordenadores actuales pueden
trabajar cómodamente con muchos procesos ejecutándose simultáneamente
y poseen sistemas de almacenamiento rápido que pueden usarse para
intercambiar información. Implementar un algoritmo que funcione de
forma concurrente es, por lo tanto, tan simple como ejecutar varios
procesos simultáneamente y que intercambien información a través de un
directorio especialmente designado para ello. La eficiencia de la
implementación no tiene por qué ser grande (y dependerá sobre todo del
tipo de procesador, del número de núcleos que posea y de la velocidad
y eficiencia del sistema de ficheros) pero, sin embargo, la
simplicidad en la programación es tal que puede compensar la menor
ganancia en velocidad obtenida de esta forma. 

Simultáneamente, está cada vez más vigente el uso de infraestructuras
{\em nube} que permiten usar desde un ordenador conectado a la red
diferentes recursos tales como CPUs virtuales o discos duros
virtuales. El hecho de que sean {\em virtuales} implica que aparezcan,
desde el punto de vista del interfaz de programación, como si se
tratara de otras infraestructuras disponibles desde el sistema
operativo. En la práctica, podemos usar un disco duro remoto situado
en la nube como si se tratara de un disco duro local. De esta forma,
también simple y transparente al programador podemos paralelizar un
algoritmo, simplemente usando una infraestructura de almacenamiento
virtual. A la vez, en algunos casos estas infraestructuras son
gratuitas, bien por el hecho de que formen parte de la misma
organización (discos conectados a la red, NAS o bien infraestructuras
creadas con OpenStack u OpenNebula) o bien porque se trate de
productos comerciales que poseen una versión gratuita, como se trata
de Dropbox, Ubuntu One u otros. 

En este trabajo mostramos la primera aproximación al uso de
infraestructuras virtuales para la creación de experimentos de
computación distribuida de forma gratuita. Estos experimentos los
aplicaremos a un tipo de algoritmo denominado algoritmo genético. 

Los algoritmos genéticos \cite{guervos2010informatica} son métodos de búsqueda y optimización
inspirados en la selección natural propuesta por
Darwin. Un algoritmo evolutivo codifica un problema en unas
estructuras de datos generalmente denominadas {\em cromosomas} y que
son una representación informática de los parámetros necesarios para
resolverlo; por ejemplo, resolver el recorrido del viajante implicaría
codificar en un {\em cromosoma} la lista de las ciudades a visitar; un
problema que tuviera varios números reales como parámetros usaría una
representación binaria con una precisión determinada de tales
parámetros, esta representación binaria es la que se usa en muchos
problemas numéricos y será la que usemos aquí por simple
conveniencia. 


\section{Estado del arte}

\section{Experimentos}

\section{Conclusiones}

\section{Agradecimientos}

\nocite{*}
\bibliographystyle{plain}
\bibliography{referencias}

\end{document}
